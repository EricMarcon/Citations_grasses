% Options for packages loaded elsewhere
\PassOptionsToPackage{unicode}{hyperref}
\PassOptionsToPackage{hyphens}{url}
%
\documentclass[
]{article}
\usepackage{amsmath,amssymb}
\usepackage{iftex}
\ifPDFTeX
  \usepackage[T1]{fontenc}
  \usepackage[utf8]{inputenc}
  \usepackage{textcomp} % provide euro and other symbols
\else % if luatex or xetex
  \usepackage{unicode-math} % this also loads fontspec
  \defaultfontfeatures{Scale=MatchLowercase}
  \defaultfontfeatures[\rmfamily]{Ligatures=TeX,Scale=1}
\fi
\usepackage{lmodern}
\ifPDFTeX\else
  % xetex/luatex font selection
\fi
% Use upquote if available, for straight quotes in verbatim environments
\IfFileExists{upquote.sty}{\usepackage{upquote}}{}
\IfFileExists{microtype.sty}{% use microtype if available
  \usepackage[]{microtype}
  \UseMicrotypeSet[protrusion]{basicmath} % disable protrusion for tt fonts
}{}
\makeatletter
\@ifundefined{KOMAClassName}{% if non-KOMA class
  \IfFileExists{parskip.sty}{%
    \usepackage{parskip}
  }{% else
    \setlength{\parindent}{0pt}
    \setlength{\parskip}{6pt plus 2pt minus 1pt}}
}{% if KOMA class
  \KOMAoptions{parskip=half}}
\makeatother
\usepackage{xcolor}
\usepackage[margin=1in]{geometry}
\usepackage{graphicx}
\makeatletter
\def\maxwidth{\ifdim\Gin@nat@width>\linewidth\linewidth\else\Gin@nat@width\fi}
\def\maxheight{\ifdim\Gin@nat@height>\textheight\textheight\else\Gin@nat@height\fi}
\makeatother
% Scale images if necessary, so that they will not overflow the page
% margins by default, and it is still possible to overwrite the defaults
% using explicit options in \includegraphics[width, height, ...]{}
\setkeys{Gin}{width=\maxwidth,height=\maxheight,keepaspectratio}
% Set default figure placement to htbp
\makeatletter
\def\fps@figure{htbp}
\makeatother
\setlength{\emergencystretch}{3em} % prevent overfull lines
\providecommand{\tightlist}{%
  \setlength{\itemsep}{0pt}\setlength{\parskip}{0pt}}
\setcounter{secnumdepth}{-\maxdimen} % remove section numbering
% definitions for citeproc citations
\NewDocumentCommand\citeproctext{}{}
\NewDocumentCommand\citeproc{mm}{%
  \begingroup\def\citeproctext{#2}\cite{#1}\endgroup}
\makeatletter
 % allow citations to break across lines
 \let\@cite@ofmt\@firstofone
 % avoid brackets around text for \cite:
 \def\@biblabel#1{}
 \def\@cite#1#2{{#1\if@tempswa , #2\fi}}
\makeatother
\newlength{\cslhangindent}
\setlength{\cslhangindent}{1.5em}
\newlength{\csllabelwidth}
\setlength{\csllabelwidth}{3em}
\newenvironment{CSLReferences}[2] % #1 hanging-indent, #2 entry-spacing
 {\begin{list}{}{%
  \setlength{\itemindent}{0pt}
  \setlength{\leftmargin}{0pt}
  \setlength{\parsep}{0pt}
  % turn on hanging indent if param 1 is 1
  \ifodd #1
   \setlength{\leftmargin}{\cslhangindent}
   \setlength{\itemindent}{-1\cslhangindent}
  \fi
  % set entry spacing
  \setlength{\itemsep}{#2\baselineskip}}}
 {\end{list}}
\usepackage{calc}
\newcommand{\CSLBlock}[1]{\hfill\break\parbox[t]{\linewidth}{\strut\ignorespaces#1\strut}}
\newcommand{\CSLLeftMargin}[1]{\parbox[t]{\csllabelwidth}{\strut#1\strut}}
\newcommand{\CSLRightInline}[1]{\parbox[t]{\linewidth - \csllabelwidth}{\strut#1\strut}}
\newcommand{\CSLIndent}[1]{\hspace{\cslhangindent}#1}
\ifLuaTeX
  \usepackage{selnolig}  % disable illegal ligatures
\fi
\usepackage{bookmark}
\IfFileExists{xurl.sty}{\usepackage{xurl}}{} % add URL line breaks if available
\urlstyle{same}
\hypersetup{
  pdftitle={Citations personnelles en gras},
  hidelinks,
  pdfcreator={LaTeX via pandoc}}

\title{Citations personnelles en gras}
\author{}
\date{\vspace{-2.5em}}

\begin{document}
\maketitle

Ce test utilise deux citations \textbf{(Abe and Rajagopal 2001}; Acs,
Audretsch, and Feldman 1991; Adelman 1969) séparée par un \texttt{+}.
Attention: le mode visuel de R Studio modifie le code: c'est un
problème, il ne faut surtout pas passer en mode visuel!

Autres cas:

\begin{itemize}
\tightlist
\item
  une citation grasse à la fin: (Abe and Rajagopal 2001; \textbf{Acs,
  Audretsch, and Feldman 1991)}.
\item
  une citation grasse au milieu : (Abe and Rajagopal 2001; \textbf{Acs,
  Audretsch, and Feldman 1991}; Adelman 1969).
\end{itemize}

Le script \texttt{post-treatment.R} permet de remplacer les parenthèses
en trop par un point-virgule. Il laisse les parenthèses de début ou de
fin en gras, ce qui est imparfait, mais il n'y a pas de moyen simple de
reconnaître les parenthèses de fin de citation.

Pour la version PDF, il faut conserver le fichier TeX
(\texttt{keep\_tex:\ true}), exécuter le script de post-traitement, puis
ouvrir le fichier TeX final et enfin le compiler.

Remarque : le filtre LUA formate correctement la ponctuation en français
dans le fichier HTML.

\phantomsection\label{refs}
\begin{CSLReferences}{1}{0}
\bibitem[\citeproctext]{ref-Abe2001}
Abe, Sumiyoshi, and A. K. Rajagopal. 2001. {``Nonadditive Conditional
Entropy and Its Significance for Local Realism.''} \emph{Physica
A-Statistical Mechanics and Its Applications} 289 (1-2): 157164.
\url{https://doi.org/10.1016/S0378-4371(00)00476-3}.

\bibitem[\citeproctext]{ref-Acs1991}
Acs, Zoltan J., David B. Audretsch, and Maryann P. Feldman. 1991.
{``Real Effects of Academic Research: Comment.''} \emph{The American
Economic Review} 82 (1): 363367.
\url{http://www.jstor.org/stable/2117624}.

\bibitem[\citeproctext]{ref-Adelman1969}
Adelman, M. A. 1969. {``Comment on the {"}h{"} Concentration Measure as
a Numbers-Equivalent.''} \emph{The Review of Economics and Statistics}
51 (1): 99101. \url{https://doi.org/10.2307/1926955}.

\end{CSLReferences}

\end{document}
